\section{Materiales y métodos}
\begin{document}

\subsection{Obtención de datos en formato String}
En este estudio debemos realizar una ampliación de la red de proteínas humanas unidas a proteínas víricas.
Esta ampliación debe hacerse usando una red de string, que emplea código ensamble para nombrar a las proteínas.
Nuestros datos están codificados con symbol o Uniprot. Por tanto, debemos hacer un cambio de esta codificación a ensamble.
Para esto hemos empleado tres métodos, y nos quedaremos con aquel que nos permita trabajar con el mayor numero de genes humanos unidos a los genes del covid.
Los métodos empleados son: bitR, biomaRt, y utilizar una tabla obtenida de la base de datos de uniprot. 

\subsection{Obtención del grafo de relaciones proteinascovid-proteinasHumanas}
Para la realización del grafo con una única componente conexa, donde se forme la red mínima del interactoma humano
que permita conectar todas las proteínas víricas del SARS-CoV2, hemos usado las funciones all_simple_paths y all_shortest_paths 
del paquete igraph de R. Debido al gran tamaño de la red y a las miles de combinaciones posibles, obteniamos caminos enormes que
se han abordado realizando pequeñas modificaciones que nos permita obtener la componente conexa con un tiempo de ejecución y 
uso de memoria razonable. 

\subsection{Obtencion de la modularidad}
Para estudiar la modularidad del grafo obtenido hemos utilizado únicamente el paquete iGraph, ya que su combinación de funiones nos permite obtener justo lo que necesitamos.
En este caso, las funiones principales son cluster_walktrap y modularity. Cluster_walktrap encontrará los subgrafos o comunidades en nuestro grafo a traves de recorridos aleatorios, y modulatiry calculará la modularidad a partir de ellos. 
A parte, la funión communities me ha permitido ver claramente los grupos, y posteriormente representarlos con plot.igraph.

\end{document}
