\section{Materiales y métodos}


\subsection{Obtención de datos en formato String}
En este estudio debemos realizar una ampliacion de la red de proteinas humanas unidas a proteinas viricas.
Esta ampliacion debe hacerse usando una red de string, que emplea codigo ensamble para nombrar a las proteinas.
Nuestros datos están codificados con symbol o Uniprot. Por tanto, debemos hacer un cambio de esta codificacion a ensamble.
Para esto hemos empleado tres metodos, y nos quedaremos con aquel que nos permita trabajar con el mayor numero de genes humanos unidos a los genes del covid.
Los métodos empleados son: bitR, biomaRt, y utilizar una tabla obtenida de la base de datos de uniprot. 

\subsection{Obtención del grafo de relaciones proteinascovid-proteinasHumanas}
Antes de abordar un problema con la gran cantidad de datos que tiene (11 millones de datos el proteoma humano), hemos creado el codigo JugarRedes.R, donde hemos creado dos dataframe con pocos datos, y dos objetos igraph para poder ir probando las diferentes funciones y formas de obtener el grafo. Una vez familiarizados con las funciones y trabajar con listas hemos pasado a la implementacion del grafo de unica componente conexa.
Para la realizacion del grafo con una unica componente conexa, donde se forme la red minima del interactoma humano
que permita conectar todas las proteínas viricas del SARS-CoV2, hemos usado las funciones all\_simple\_paths y all\_shortest\_paths 
del paquete igraph de R. Debido al gran tamaño de la red y a las miles de combinaciones posibles, obteniamos caminos enormes que
se han abordado realizando pequeñas modificaciones que nos permita obtener la componente conexa con un tiempo de ejecucion y 
uso de memoria razonable. 

\subsection{Comparacion entre la red obtenida y la red humana}
Usando funciones del paquete igraph se han analizado ciertas propiedades de ambas redes para poder compararlas. 
Las propiedades analizadas son: densidad, reprocidad, diametro, grado, distribucion de grado, distancia media y asortatividad.

\subsection{Obtencion de la modularidad}
Para estudiar la modularidad del grafo obtenido hemos utilizado unicamente el paquete iGraph, ya que su combinacion de funciones nos permite obtener justo lo que necesitamos.
En este caso, las funciones principales son cluster\_walktrap y modularity. Cluster\_walktrap encontrara los subgrafos o comunidades en nuestro grafo a traves de recorridos aleatorios, y modulatiry calculara la modularidad a partir de ellos. 
A parte, la funcion communities me ha permitido ver claramente los grupos, y posteriormente representarlos con plot.igraph.

\subsection{Obtencion de la centralidad}
La centralidad determina la importancia que puede obtener un nodo dentro de una red, sin embargo, esta importancia puede medirse de numerosas formas, siendo algunas de ellas m\'as correctas en ocasiones que otras.
Con el objetivo de estudiar la centralidad de nuestra red, se ha decidido emplear los paquetes \textbf{igraph} y \textbf{CINNA} principalmente. Adem\'as, se ha utilizado un sistema m\'as simple pero que forma una sola componente conexa, debido al alto costo de computaci\'on que tiene el formar una sola componente conexa con todos los nodos que tenemos.
Mediante el paquete CINNA hemos decidido comprobar la centralidad que se obtiene mediante algunos de los metodos mas comunes como pueden ser el algoritmo \textbf{page-rank}, centralidad mediante el grado de cada \textbf{nodo}, la centralidad basada en la \textbf{cercania} y la centralidad a traves del \textbf{betweeness}(capacidad para estar en medio de los paths biologicos importantes).
Para obtener el m\'etodo de centralidad que sea considerado m\'as adecuado en nuestro caso, realizaremos una \textbf{pca}(an\'alisis de componentes) en la que se ver\'a reflejada la participaci\'on de cada medida de modularidad en nuestros datos. De esta manera podremos coger el método más adecuado para estos datos.

\subsection{Obtencion de la robustez}
Para obtener una medida de la robustez de nuetro grafo hemos realizado dos ataques dirigido. Para ambos hemos usado la funcion robustness del paquete brainGraph, la unica variacion es el parametro que determina el tipo de ataque, dado que realizamos un ataque basado en degree y otro en betweenness. Luego las hemos representado con un plot para una mejor visualizacion.
