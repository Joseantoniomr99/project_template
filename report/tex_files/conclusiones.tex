\section{Conclusiones}
Para este trabajo de investigacion conseguimos obtener un grafo de las cualidades deseadas, quizas no contiene los caminos mas optimos, pero debido al gran tiempo y memoria de ejecucion se ha simplificado hasta poder obtener uno lo suficientemente sencillo y completo. Este nos permite obtener, mediante cluster profiler, las funciones en las que intervienen nuestras proteinas y las componentes de las que forman parte. Tras el analisis realizado, vemos que obtenemos caracteristicas de las redes reales y no de redes aleatorias. 

A partir de este grafo hemos podido obtener mucha información que podria ser relevante a la hora de expandir en esta investigación, como podría ser la alta modularidad observada.
Esto podriamos haberlo supuesto por el contexto del proyecto, pero un resultado inesperado fue lo efectivo que seria un ataque dirigido mediante betweenness. El breakdown point se encuentra al incio del ataque, tan solo en el tercer ataque la red se deshace en un 90 porciento. Esto podria deberse a la presencia de una o varias proteinas esenciales para gran parte de los procesos de la red, y por tanto al eliminarla bloqueamos dichos procesos.

-Hablar de centralidad y enriquecimiento brevemente-
