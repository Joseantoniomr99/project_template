
\section{Resultados}
\begin{document}

\subsection{Obtención de datos en formato String}
Se utilizan los tres métodos mencionados en la sección de Materiales y Métodos.

\subsubsection{bitR}


\subsubsection{biomaRt}
Para este análisis hemos utilizado un paquete de Rstudio llamado biomaRt. Este paquete nos permite cambiar la codificación de uniprot a ensamble.
El código perteneciente a este análisis se encuentra en la carpeta code y se denomina biomart.R. En este código viene detallado con comentarios el
proceso que se ha llevado a cabo. Básicamente los pasos seguidos son: 
- pasar de código uniprot a ensamble
- comprobar cuantos de estos códigos ensamble se encuentran en el proteoma completo, guardamos en un dataframe los valores ensamble y uniprot.  Comprobamos que solo nos quedamos de 332 proteinas humanas unidas a viricas con 107.
- Hacemos un bucle y guardamos en un dataframe los valores de uniprot y ensamble aquellos que coinciden con nuestra tabla de entrada.
- Hacemos un merge que nos une los dos dataframe, el de entrada con genes covid y humanos, y el obtenido con el cambio de uniprot a ensamble. 


\subsubsection{Tabla UniProt}
Para este análisis hemos utilizado una tabla obtenida de la base de datos de UniProt, que contiene tres columnas. Un valor de uniprot, uno de ensamble y otro tipo de código uniprot para el mismo gen. 
El código perteneciente a este análisis se encuentra en la carpeta code y se denomina tablaObtenidaUniprot.R. Nuevamente en el código vienen detallados los pasos seguidos. 
El proceso llevado a cabo es el siguiente:
- Leemos los ficheros: la tabla de uniprot, la tabla de relaciones entre genes viricos y humanos, y el proteoma de interacción completo. El fichero con la tabla de uniprot tiene filas que no contienen código ensamble para algunos de los genes. Así que realizaremos un filtrado que elimine estas filas.
- Una vez realizado esto, buscaremos cuantos de estos codigos ensamble se encuentran en el proteoma completo. Vemos que solo perdemos cuatro de estos códigos.
- Posteriormente, buscaremos cuantos de los codigos uniprot de mi tabla de entrada podemos convertir a ensamble. 


\subsection{Obtención del grafo de relaciones proteinascovid-proteinasHumanas}

Para abordar el problema de la creación de un grafo conexo, nos hemos enfrentado a tiempos de ejecución elevados y un gasto de memoria elevada. 
El código denominado AlgoritmoRedCompleta.R muestra como se ha llevado a cabo el proceso de obtención de red. 
Una vez obtenido en el apartado anterior la tabla con nuestros genes uniprot en formato string, ya podríamos usar funciones pertenecientes a igraph que buscaran rutas entre una y otra proteina humana dentro del proteoma. 
Como primer paso se intento obtener todas las rutas posibles entre todas las proteinas humanas unidas al covid, usando la función all_simple_paths. 
Nos enfrentamos a una compilación donde tras 15 horas de ejecución continua nuestro código no había terminado de definir las rutas entre las proteinas,además de haber generado un archivo de casi 15 gigas que por poco ocupaba toda la memoria de R. 
Decidimos hacer una pequeña modificaciones y calcular todas las rutas posibles con la misma función pero simplemente haciendo las combinaciones de un gen con su siguiente. Nuevamente el tiempo de ejecución y la memoria ocupada eran inviables. 
También se investigo el parametro cuttof que permitia establecer un tamaño minimo del camino, pero tampoco conseguiamos un resultado que su tiempo de ejecucion fuera factible. 
Asi que, tras varios dias de prueba, decidimos usar la función all_shortest_path, que obtenia la ruta mas rapida de una proteina a otra. Creíamos que esto sería más rápido y obtendriamos los resultados necesarios para poder obtener el grafo. Pero nuevamente, vimos que el tiempo y la memoria que se ocupaba eran enormes. Otra opción era comparar una proteina con su siguiente, pero esto tras 5 horas y 8 gigas no habia terminado de compilar. 
Como última opción, decidimos hacer una simplificación. Cogimos 2 genes humanos por cada gen de covid, e hicimos un dataframe. Este dataframe es el que usamos para obtener el grafo completo. 
Probamos dos opciones, una de ellas utilizaba todas las combinaciones posibles entre estas dos proteinas de cada gen (51 proteinas en total) y la otra utilizaba una proteina con su siguiente. Para optimizacion de tiempo y memoria usamos la segunda de las opciones, obteniendo 100 elementos donde cada una tenia varias listas de uniones de grafos. Como muchas de ellas se repetian usamos unique y conseguimos reducir estas repeticiones. Obteniendo 18012 rutas, que seguiamn teniendo elementos repetidos. A continuación, creamos dos vectores, y hacemos un bucle que vaya recorriendo estas rutas, y me vaya añadiente una proteina en un vector, y la proteina con la que interacciona en el siguiente. Todo esto se integra en un data frame y usamos unique para eliminar las repeticiones. 
Este dataframe lo convertimos en objeto igraph y mediante %u% conseguimos unir dos objetos igraph, esto unirá los dos objetos igraph y ya podremos comprobar si hemos obtenido o no la componete conexa usando la función components. Es importante mencionar que para poder crear las rutas, nuestro grafo de proteoma debe tener una unica componente conexa, y si tiene mas de una, las proteinas buscadas pertenezcan al mismo. El resultado se comentará en el apartado de discusiones. 




















\end{document}
