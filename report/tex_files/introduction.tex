\section{Introducción}
\begin{document}
Tal y como nos muestra la biología de sístemas, sabemos que un organismo biológico es un sistema integrado e iterrelacionado de genes, proteínas y reacciones bioqímicas, que da lugar a procesos biológicos. Nuestro estudio consiste en ver la relación de los genes viricos pertenecientes al SARS-CoV2 con los genes humanos. Partimos de alrededor de 30 grafos individuales, que muestra la proteina virica y los patógenos unidos a las proteínas humanas. Estos datos al ser descargados nos ofrecen seis ficheros .xslx que muestra el nombre en uniprot, symbol de estas proteinas humanas, junto con algunas descripciones. Nuestros ficheros de interés son Network_table.xlsx y Prey_Lookup_Table.xlsx, analizaremos estos datos y veremos cuál nos será útil para el trabajo que queremos llevar a cabo. Nuestro objetivo es tomar cada uno de estos grafos y unirlos entre sí mediante interacciones proteínahumana-proteinahumana, para conseguir un único grafo conexo completo. Para esto usaremos la red del proteoma humano conseguida en string "9606.protein.links.v11.5.txt".
\end{document}
