\section{Discusión}
\begin{Document}

\subsection{Datos a emplear}
Tras comparar los resultados de los tres algoritmos anteriores. Vemos que la tabla final que deberemos usar para el análisis se obtiene usando el tercer método, donde no perdemos ninguna proteina virica y unicamente perdemos 4 de las humanas. 
Ya que usando el método de biomaRt perdemos 225 genes humanos, perdemos tambien 4 genes viricos pertenecientes al Covid. 

\subsection{Grafo conexo}
Después de todos los intentos por obtener un grafo conexo, mencionados en el apartado de resultados. Hemos conseguido obtener el grafo conexo, puede que no sea el grafo óptimo ni contenga todas las rutas, debido a los diferentes problemas vistos, pero hemos conseguido unir mediante rutas de proteinas humanas, las proteinas del Covid. 
Hemos podido comprobar que el grafo del proteoma total formaba una única componente conexa, por tanto todas las proteinas se encuentran en el mismo lugar, y pueden usarse las funciones de igraph para conseguir las rutas. 
También se podría haber utilizado las matrices de adyacencia para obtener rutas, tal y como se vio en clase de teoría, pero preferimos haber optado por el metodo del uso de funciones del paquete igraph. La función empleada para ver si el grafo obtenido es conexo es components, donde su parametro no indica el numero de componentes, y su parametro csize indica el número de vertices diferentes. 
Por tanto, obtenemos una red formada por 2888 nodos, 9333 enlaces entre ellos y una única compoente conexa. De la cual podremos analizar su modularidad y analisis funcional.
\end{Document}
