\documentclass{bmcart}

%%%%%%%%%%%%%%%%%%%%%%%%%%%%%%%%%%%%%%%%%%%%%%
%%                                          %%
%% CARGA DE PAQUETES DE LATEX               %%
%%                                          %%
%%%%%%%%%%%%%%%%%%%%%%%%%%%%%%%%%%%%%%%%%%%%%%

%%% Load packages
\usepackage{amsthm,amsmath}
\usepackage{graphicx}
%\RequirePackage[numbers]{natbib}
%\RequirePackage{hyperref}
\usepackage[utf8]{inputenc} %unicode support
%\usepackage[applemac]{inputenc} %applemac support if unicode package fails
%\usepackage[latin1]{inputenc} %UNIX support if unicode package fails


%%%%%%%%%%%%%%%%%%%%%%%%%%%%%%%%%%%%%%%%%%%%%%
%%                                          %%
%% COMIENZO DEL DOCUMENTO                   %%
%%                                          %%
%%%%%%%%%%%%%%%%%%%%%%%%%%%%%%%%%%%%%%%%%%%%%%

\begin{document}

	\begin{frontmatter}
	
		\begin{fmbox}
			\dochead{Research}
			
			%%%%%%%%%%%%%%%%%%%%%%%%%%%%%%%%%%%%%%%%%%%%%%
			%% INTRODUCIR TITULO PROYECTO               %%
			%%%%%%%%%%%%%%%%%%%%%%%%%%%%%%%%%%%%%%%%%%%%%%
			
			\title{Interaccion a nivel molecular hospedador-virus del SARS-CoV2}
			
			%%%%%%%%%%%%%%%%%%%%%%%%%%%%%%%%%%%%%%%%%%%%%%
			%% AUTORES. METER UNA ENTRADA AUTHOR        %%
			%% POR PERSONA                              %%
			%%%%%%%%%%%%%%%%%%%%%%%%%%%%%%%%%%%%%%%%%%%%%%
			
			\author[
			  addressref={aff1},                   % ESTA LINEA SE COPIA IGUAL PARA CADA AUTOR
			  corref={aff1},                       % ESTA LINEA SOLO DEBE TENERLA EL COORDINADOR DEL GRUPO
			  email={emanuelamariastoia@gmail.com}   % VUESTRO CORREO ACTIVO
			]{\inits{E.M.S}\fnm{Emanuela Maria} \snm{Stoia}} % inits: INICIALES DE AUTOR, fnm: NOMBRE DE AUTOR, snm: APELLIDOS DE AUTOR
			\author[
			  addressref={aff1},
			  email={joseantoniomr99@gmail.com}
			]{\inits{J.A.M}\fnm{Jose Antonio} \snm{Muñoz}}
			\author[
			  addressref={aff1},
			  email={stephancharles1999@yahoo.es}
			]{\inits{S.C.N}\fnm{Stephan Charles} \snm{Nielson}}
			
			%%%%%%%%%%%%%%%%%%%%%%%%%%%%%%%%%%%%%%%%%%%%%%
			%% AFILIACION. NO TOCAR                     %%
			%%%%%%%%%%%%%%%%%%%%%%%%%%%%%%%%%%%%%%%%%%%%%%
			
			\address[id=aff1]{%                           % unique id
			  \orgdiv{ETSI Informática},             % department, if any
			  \orgname{Universidad de Málaga},          % university, etc
			  \city{Málaga},                              % city
			  \cny{España}                                    % country
			}
		
		\end{fmbox}% comment this for two column layout
		
		\begin{abstractbox}
		
			\begin{abstract} % abstract
			
			%%%%%%%%%%%%%%%%%%%%%%%%%%%%%%%%%%%%%%%%%%%%%%%
			%% RESUMEN BREVE DE NO MAS DE 100 PALABRAS   %%
			%%%%%%%%%%%%%%%%%%%%%%%%%%%%%%%%%%%%%%%%%%%%%%%	
			
			En este proyecto de investigacion se lleva a cabo una expansion en red, de una red de interaccion entre proteinas de Covid y proteinas 				humanas. 
			Esta expansion se llevara acabo buscando las rutas entre proteínas humanas usando el paquete de igraph. Esto es util para la 					biología se sistemas, donde vimos no era suficiente un analisis reduccionista sino una vision holistica.
			Esta vision nos permitira ver la influencia del covid en un organismo humano viendo todas las proteinas afectadas, y no solamente los 				primeros enlaces. Pudiendo hacer un analisis funcional para ver las funciones a las que afecta. 
			
			
			\end{abstract}
			
			%%%%%%%%%%%%%%%%%%%%%%%%%%%%%%%%%%%%%%%%%%%%%%
			%% PALABRAS CLAVE DEL PROYECTO              %%
			%%%%%%%%%%%%%%%%%%%%%%%%%%%%%%%%%%%%%%%%%%%%%%
			
			\begin{keyword}
			\kwd{rutas}
			\kwd{proteinas}
			\kwd{nodos}
			\kwd{red}
			\kwd{interaccino}
			\kwd{covid}
			\kwd{proteoma}
			\kwd{grafo}
			\kwd{enlaces}
			\end{keyword}
		
		
		\end{abstractbox}
	
	\end{frontmatter}	
	
	%%%%%%%%%%%%%%%%%%%%%%%%%%%%%%%%%
	%% COMIENZO DEL DOCUMENTO REAL %%
	%%%%%%%%%%%%%%%%%%%%%%%%%%%%%%%%%
	
	\section{Introducción}

Tal y como nos muestra la biolog\'ia de sistemas, sabemos que un organismo biol\'ogico es un sistema integrado e interrelacionado de genes,
prote\'inas y reacciones bioqu\'imicas, que da lugar a procesos biol\'ogicos. Nuestro estudio consiste en ver la relación de los genes viricos 
pertenecientes al SARS-CoV2 con los genes humanos. Partimos de alrededor de 30 grafos individuales, que muestra la prote\'ina virica y los
pat\'ogenos unidos a las prote\'inas humanas. Estos datos al ser descargados nos ofrecen seis ficheros .xslx que muestra el nombre en uniprot, 
symbol de estas proteinas humanas, junto con algunas descripciones. Nuestros ficheros de inter\'es son Network\_table.xlsx y Prey\_Lookup\_Table.xlsx, 
analizaremos estos datos y veremos cu\'al nos ser\'a \'util para el trabajo que queremos llevar a cabo. Nuestro objetivo es tomar cada uno de estos grafos
y unirlos entre s\'i mediante interacciones proteínahumana-proteinahumana, para conseguir un \'unico grafo conexo completo.
Para esto usaremos la red del proteoma humano conseguida en string "9606.protein.links.v11.5.txt".

	\section{Materiales y métodos}


\subsection{Obtención de datos en formato String}
En este estudio debemos realizar una ampliacion de la red de proteinas humanas unidas a proteinas viricas.
Esta ampliacion debe hacerse usando una red de string, que emplea codigo ensamble para nombrar a las proteinas.
Nuestros datos están codificados con symbol o Uniprot. Por tanto, debemos hacer un cambio de esta codificacion a ensamble.
Para esto hemos empleado tres metodos, y nos quedaremos con aquel que nos permita trabajar con el mayor numero de genes humanos unidos a los genes del covid.
Los métodos empleados son: bitR, biomaRt, y utilizar una tabla obtenida de la base de datos de uniprot. 

\subsection{Obtención del grafo de relaciones proteinascovid-proteinasHumanas}
Para la realizacion del grafo con una unica componente conexa, donde se forme la red minima del interactoma humano
que permita conectar todas las proteínas viricas del SARS-CoV2, hemos usado las funciones all\_simple\_paths y all\_shortest\_paths 
del paquete igraph de R. Debido al gran tamaño de la red y a las miles de combinaciones posibles, obteniamos caminos enormes que
se han abordado realizando pequeñas modificaciones que nos permita obtener la componente conexa con un tiempo de ejecucion y 
uso de memoria razonable. 

\subsection{Obtencion de la modularidad}
Para estudiar la modularidad del grafo obtenido hemos utilizado unicamente el paquete iGraph, ya que su combinacion de funciones nos permite obtener justo lo que necesitamos.
En este caso, las funciones principales son cluster\_walktrap y modularity. Cluster\_walktrap encontrara los subgrafos o comunidades en nuestro grafo a traves de recorridos aleatorios, y modulatiry calculara la modularidad a partir de ellos. 
A parte, la funcion communities me ha permitido ver claramente los grupos, y posteriormente representarlos con plot.igraph.

\subsection{Obtencion de la centralidad}
La centralidad determina la importancia que puede obtener un nodo dentro de una red.
Con el objetivo de estudiar la centralidad de nuestra red, se ha decidido emplear los paquetes \textbf{igraph} y \textbf{CINNA} principalmente.
Mediante el paquete CINNA hemos decidido comprobar la centralidad que se obtiene mediante algunos de los metodos mas comunes como pueden ser el algoritmo \textbf{page-rank}, centralidad mediante el grado de cada \textbf{nodo}, la centralidad basada en la \textbf{cercania} y la centralidad a traves del \textbf{betweeness}(capacidad para estar en medio de los paths biologicos importantes).
Se ha escogido el algoritmo closeness ya que se ha visto como el más eficiente según la pca.

	
\section{Resultados}
\begin{document}

\subsection{Obtención de datos en formato String}
Se utilizan los tres métodos mencionados en la sección de Materiales y Métodos.

\subsubsection{bitR}
Para este análisis hemos usado el paquete bitR de Rstudio. Hemos realizado un bitR del proteoma humano para un score maayor de 650. Posteriormente hemos comprobado cuantas de las proteinas humanas de nuestra tabla de unión con covid se encuentra en nuestro proteoma obtenido. Posteriormente, hemos comprobado cuantas de ellas están unidas a genes covid, y cuantos genes covid son.

\subsubsection{biomaRt}
Para este análisis hemos utilizado un paquete de Rstudio llamado biomaRt. Este paquete nos permite cambiar la codificación de uniprot a ensamble.
El código perteneciente a este análisis se encuentra en la carpeta code y se denomina biomart.R. En este código viene detallado con comentarios el
proceso que se ha llevado a cabo. Básicamente los pasos seguidos son: 
- pasar de código uniprot a ensamble
- comprobar cuantos de estos códigos ensamble se encuentran en el proteoma completo, guardamos en un dataframe los valores ensamble y uniprot.  Comprobamos que solo nos quedamos de 332 proteinas humanas unidas a viricas con 107.
- Hacemos un bucle y guardamos en un dataframe los valores de uniprot y ensamble aquellos que coinciden con nuestra tabla de entrada.
- Hacemos un merge que nos une los dos dataframe, el de entrada con genes covid y humanos, y el obtenido con el cambio de uniprot a ensamble. 


\subsubsection{Tabla UniProt}
Para este análisis hemos utilizado una tabla obtenida de la base de datos de UniProt, que contiene tres columnas. Un valor de uniprot, uno de ensamble y otro tipo de código uniprot para el mismo gen. 
El código perteneciente a este análisis se encuentra en la carpeta code y se denomina tablaObtenidaUniprot.R. Nuevamente en el código vienen detallados los pasos seguidos. 
El proceso llevado a cabo es el siguiente:
- Leemos los ficheros: la tabla de uniprot, la tabla de relaciones entre genes viricos y humanos, y el proteoma de interacción completo. El fichero con la tabla de uniprot tiene filas que no contienen código ensamble para algunos de los genes. Así que realizaremos un filtrado que elimine estas filas.
- Una vez realizado esto, buscaremos cuantos de estos codigos ensamble se encuentran en el proteoma completo. Vemos que solo perdemos cuatro de estos códigos.
- Posteriormente, buscaremos cuantos de los codigos uniprot de mi tabla de entrada podemos convertir a ensamble. 


\subsection{Obtención del grafo de relaciones proteinascovid-proteinasHumanas}

Para abordar el problema de la creación de un grafo conexo, nos hemos enfrentado a tiempos de ejecución elevados y un gasto de memoria elevada. 
El código denominado AlgoritmoRedCompleta.R muestra como se ha llevado a cabo el proceso de obtención de red. 
Una vez obtenido en el apartado anterior la tabla con nuestros genes uniprot en formato string, ya podríamos usar funciones pertenecientes a igraph que buscaran rutas entre una y otra proteina humana dentro del proteoma. 
Como primer paso se intento obtener todas las rutas posibles entre todas las proteinas humanas unidas al covid, usando la función all_simple_paths. 
Nos enfrentamos a una compilación donde tras 15 horas de ejecución continua nuestro código no había terminado de definir las rutas entre las proteinas,además de haber generado un archivo de casi 15 gigas que por poco ocupaba toda la memoria de R. 
Decidimos hacer una pequeña modificaciones y calcular todas las rutas posibles con la misma función pero simplemente haciendo las combinaciones de un gen con su siguiente. Nuevamente el tiempo de ejecución y la memoria ocupada eran inviables. 
También se investigo el parametro cuttof que permitia establecer un tamaño minimo del camino, pero tampoco conseguiamos un resultado que su tiempo de ejecucion fuera factible. 
Asi que, tras varios dias de prueba, decidimos usar la función all_shortest_path, que obtenia la ruta mas rapida de una proteina a otra. Creíamos que esto sería más rápido y obtendriamos los resultados necesarios para poder obtener el grafo. Pero nuevamente, vimos que el tiempo y la memoria que se ocupaba eran enormes. Otra opción era comparar una proteina con su siguiente, pero esto tras 5 horas y 8 gigas no habia terminado de compilar. 
Como última opción, decidimos hacer una simplificación. Cogimos 2 genes humanos por cada gen de covid, e hicimos un dataframe. Este dataframe es el que usamos para obtener el grafo completo. 
Probamos dos opciones, una de ellas utilizaba todas las combinaciones posibles entre estas dos proteinas de cada gen (51 proteinas en total) y la otra utilizaba una proteina con su siguiente. Para optimizacion de tiempo y memoria usamos la segunda de las opciones, obteniendo 100 elementos donde cada una tenia varias listas de uniones de grafos. Como muchas de ellas se repetian usamos unique y conseguimos reducir estas repeticiones. Obteniendo 18012 rutas, que seguiamn teniendo elementos repetidos. A continuación, creamos dos vectores, y hacemos un bucle que vaya recorriendo estas rutas, y me vaya añadiente una proteina en un vector, y la proteina con la que interacciona en el siguiente. Todo esto se integra en un data frame y usamos unique para eliminar las repeticiones. 
Este dataframe lo convertimos en objeto igraph y mediante el operados de unión conseguimos unir dos objetos igraph, esto unirá los dos objetos igraph y ya podremos comprobar si hemos obtenido o no la componete conexa usando la función components. Es importante mencionar que para poder crear las rutas, nuestro grafo de proteoma debe tener una unica componente conexa, y si tiene mas de una, las proteinas buscadas pertenezcan al mismo. El resultado se comentará en el apartado de discusiones. 

\subsubsection{Modularidad}

Para obtener la modularidad del grafo hemos usado el paquete igraph. El código correspondiente se encuentra en un archivo llamado Modularidad y Centralidad.R
Lo primero fue obtener los módulos del grafo mediante la función cluster_walktrap. Obtuvimos 68 grupos de tamaño diverso, desde 4 hasta 1778. Una vez obtenido esto, se le pasa a la función modularity transformándolo a un vector de membership (con la funcion membership) y pasando el grafo original como parámetro. El resultado es una modularidad de 0.935.
Después de esto, representamos algunos valores pertinentes, como una lista de los tamaños de los grupos y Grafo en el que solo se ven los modulos coloreados.
Este grafo se hizo con plot.igraph, y nos muestra los 68 modulos por separado.
Aparte, vimos la posibilidad de representar un dendograma, pero no aporta ninguna informacion util debido a la cantidad de datos.



\end{document}

	\section{Discusión}

\subsection{Datos a emplear}
Tras comparar los resultados de los tres algoritmos anteriores. Vemos que la tabla final que deberemos usar para el analisis se obtiene usando el tercer metodo, donde no perdemos ninguna proteina virica y unicamente perdemos 4 de las humanas. 
Ya que usando el metodo de biomaRt perdemos 225 genes humanos, perdemos tambien 4 genes viricos pertenecientes al Covid. Y hemos comprobado que usando bitR solo obteniamos 18 de las 332 proteinas humanas y además unicamente dos de las proteinas covid. 

\subsection{Grafo conexo}
Despues de todos los intentos por obtener un grafo conexo, mencionados en el apartado de resultados. Hemos conseguido obtener el grafo conexo, puede que no sea el grafo optimo ni contenga todas las rutas, debido a los diferentes problemas vistos, pero hemos conseguido unir mediante rutas de proteinas humanas, las proteinas del Covid. 
Hemos podido comprobar que el grafo del proteoma total formaba una unica componente conexa, por tanto todas las proteinas se encuentran en el mismo lugar, y pueden usarse las funciones de igraph para conseguir las rutas. 
Tambien se podria haber utilizado las matrices de adyacencia para obtener rutas, tal y como se vio en clase de teoria, pero preferimos haber optado por el metodo del uso de funciones del paquete igraph. La funcion empleada para ver si el grafo obtenido es conexo es components, donde su parametro no indica el numero de componentes, y su parametro csize indica el numero de vertices diferentes. 
Por tanto, obtenemos una red formada por 2888 nodos, 9333 enlaces entre ellos y una unica compoente conexa. De la cual podremos analizar su modularidad y analisis funcional.

\subsection{Comparacion entre la red obtenida y la red humana}
Vistos los resultados de la comparación podremos comentar lo siguiente:
- Vemos que la red de proteoma humana es más densa, por tanto, contiene muchas más conexiones entre nodos que la nueva creada.
- Observamos que la red del porteoma humana tiene caminos mas cortos entre si que la red obtenida, consiguiendo asi una red menos compacta.
- Aqui observamos una de las propiedades de las redes reales donde vemos que tenemos unos pocos nodos de alto grado y muchos nodos de grado mas bajo. 
- Una característica destacable es que en la red humana los nodos con el mismo grado tiende a undirse entres si, pero en nuestra red obtenida la asortatividad es muy baja.
- La distribución de grado es más o menos similar en ambas, y no siguen una distribucion binomial que corresponde a grafo aleatorios. Sino que la frecuencia va aumentando con el grado, para que asi se puedan ir formando los denominados hubs.
\subsection{Centralidad obtenida}
Tras realizar un breve an\'alisis de la centralidad de nuestro modelo usado(hay que recordar que estamos usando un subgrafo que forma una sola componente conexa, porque todo el grafo entero como componente conexa tiene un costo computacional demasiado elevado), hemos obtenido resultados que no eran del todo esperados.\newline
Recapitulando, los valores obtenidos se han normalizado mediante la funci\'on \textbf{closeness} que obtiene la centralidad, es decir, todos los valores de centralidad est\'an comprendidos entre 0 y 1. En primer lugar, hemos obtenido una media de centralidad de 0.287, n\'umero que puede ser considerado muy bajo, dado que podr\'ia llegar hasta 1. Esto quiere decir que probablemente gran parte de nuestros nodos del subgrafo carezcan de importancia dentro de este seg\'un su posici\'on relativa en el subgrafo. Tambi\'en se podr\'ia interpretar como que hay bastantes nodos distanciados del resto de nodos en la red, por lo que la componente conexa que tenemos podr\'ia estar muy dispersa.\newline
Por un lado, no tenemos ning\'un nodo con una centralidad menor de 0.1, sin embargo, la mayor\'ia de los nodos(1990) tienen una centralidad comprendida entre 0.1 y 0.3. Esto nos ayuda a reafirmar que la mayor\'ia de los nodos de la red poseen una centralidad baja. Teniendo en cuenta que estamos midiendo la centralidad con la cercan\'ia, y sabiendo que la centralidad de cercan\'ia de un nodo se calcula como $1/(\Sigma(d(i,j)), i != j)$, tenemos que cuanto mayor sean las distancias de un nodo al resto de nodos, menor centralidad tendr\'a. Dicho de otro modo, la mayor\'ia de los nodos poseen una distancia al resto de nodos bastante elevada.\newline
Por otro lado, aunque hay unos cuantos nodos(898) con un valor de centralidad de entre 0.3 y 0.5, no hay ninguno cuyo valor de centralidad sea mayor que 0.5, lo que nos quiere decir que no hay ning\'un nodo en nuestro subgrafo que pueda ser considerado central o importante. Esto puede deberse, o bien a que el subgrafo generado no consiga representar adecuadamente al grafo conexo original, o bien, simplemente los nodos tanto de nuestro subgrafo, como del grafo original est\'an demasiado dispersos entre ellos como para considerar que la centralidad y la importancia que tienen dentro de la red es suficientemente alta.


	\section{Conclusiones}
Para este trabajo de investigacion conseguimos obtener un grafo de las cualidades deseadas, quizas no contiene los caminos mas optimos, pero debido al gran tiempo y memoria de ejecucion se ha simplificado hasta poder obtener uno lo suficientemente sencillo y completo. Este nos permite obtener, mediante cluster profiler, las funciones en las que intervienen nuestras proteinas y las componentes de las que forman parte. Tras el analisis realizado, vemos que obtenemos caracteristicas de las redes reales y no de redes aleatorias. 

A partir de este grafo hemos podido obtener mucha información que podria ser relevante a la hora de expandir en esta investigación, como podría ser la alta modularidad observada.
Esto podriamos haberlo supuesto por el contexto del proyecto, pero un resultado inesperado fue lo efectivo que seria un ataque dirigido mediante betweenness. El breakdown point se encuentra al incio del ataque, tan solo en el tercer ataque la red se deshace en un 90 porciento. Esto podria deberse a la presencia de una o varias proteinas esenciales para gran parte de los procesos de la red, y por tanto al eliminarla bloqueamos dichos procesos.

-Hablar de centralidad y enriquecimiento brevemente-

	
	
	%%%%%%%%%%%%%%%%%%%%%%%%%%%%%%%%%%%%%%%%%%%%%%
	%% OTRA INFORMACIÓN                         %%
	%%%%%%%%%%%%%%%%%%%%%%%%%%%%%%%%%%%%%%%%%%%%%%
	
	\begin{backmatter}
	
		\section*{Abreviaciones}%% if any
			Indicar lista de abreviaciones mostrando cada acrónimo a que corresponde
		
		\section*{Disponibilidad de datos y materiales}%% if any
			https://github.com/Joseantoniomr99/project\_template.git
		
		\section*{Contribución de los autores}
			E.M.S : Encargada de generar con bitR el grafo del proteoma del score 800-950, analisis de mejor forma de uso para cambiar del tipo de 
			datos de proteinas iniciales a tipo de datos string para conicidir con proteoma, obtención de datos para creacion de red, 
			practica con una red de jueguete para obtener los caminos al grado, creacion del grafo conexo de union de proteinas covid con proteinas 
			humanas, creacion del codigo que compara la red obtenida con la red del proteoma total,memoria de latex de los codigos
			anteriores mencionados, con sus partes de material y metodos resultados y discusion, memoria de la introduccion del trabajo en latex 
			y el resumen.
			
			J.A.M: Encargado de crear el algoritmo que traduzca el proteoma de Ensembl a Entrezid mediante bitr, script para obtener los datos traducidos de ensembl 			 a Uniprot, elecci\'on del mejor m\'etodo de centralidad mediante pca y an\'alisis de la centralidad obtenida con sus correspondientes gr\'aficas, 				traducci\'on del subgrafo mediante biomaRt a Entrezid junto con enriquecimiento funcional de los genes de dicho subgrafo e im\'agenes. Materiales y 				m\'etodos, resultados y discusiones del estudio de la centralidad y el enriquecimiento funcional.
			
			S.C.N : Encargado de generar con bitR el grafo del proteoma de score 650-800 y luego unir los tres grafos resultantes de los diferentes 
			scores de bitR para obtener el grafo completo. Investigación para encontrar funciones que nos permitieran relacionar las proteinas 
			relevantes dentro del proteoma humano completo (conclusion shortest.paths de iGraph). 
			Estudio de la modularidad del grafo final y representacion de los modulos resultantes. Estudio de robustez y obtencion de graficas representativas.
			Memoria de las secciones pertinentes a los codigos mencionados anteriormente.
		
		
		%%%%%%%%%%%%%%%%%%%%%%%%%%%%%%%%%%%%%%%%%%%%%%%%%%%%%%%%%%%%%%%%%%%%%%%%%%%%%%%%%%%%%%%%
		%% BIBLIOGRAFIA: no teneis que tocar nada, solo sustituir el archivo bibliography.bib %%
		%% por el que hayais generado vosotros                                                %%
		%%%%%%%%%%%%%%%%%%%%%%%%%%%%%%%%%%%%%%%%%%%%%%%%%%%%%%%%%%%%%%%%%%%%%%%%%%%%%%%%%%%%%%%%
		
		\bibliographystyle{bmc-mathphys} % Style BST file (bmc-mathphys, vancouver, spbasic).
		\bibliography{bibliography}      % Bibliography file (usually '*.bib' )
	
	\end{backmatter}
\end{document}
